\section{Introduction}
Nowadays, traffic management stands as a critical challenge, particularly at bustling intersection where the flow of vehicles varies significantly throughout the day. The intersection in front of the West gate of University Tunku Abdul Rahman represents one such critical juncture that experiences fluctuating traffic volumes, primarily bustling with activity before and after classes. Traditional traffic light systems, which operate on fixed time intervals, often fall short in efficiently managing these variations in traffic flow, leading to unnecessary delays during off-peak hours and potential congestion during peak times.

Recognizing the need for a more adaptive approach to traffic management, this project proposes the development and implementation of a smart traffic light controller system. Leveraging advanced sensor technology, the proposed system aims to dynamically adjust the traffic light phases in real-time based on actual traffic conditions at the intersection. Sensors installed at the approach of each direction will provide immediate feedback on the presence or absence of vehicles, enabling the traffic light to intelligently decide when to change phases, thus optimizing traffic flow and reducing wait times.

The core of this assignment is the design and implementation of the traffic light controller using SystemVerilog, a \ac{hdl} renowned for its efficiency and flexibility in modelling complex digital systems. This language facilitates a high level of abstraction in design, allowing for a comprehensive simulation and verification of the traffic light controller's functionality before physical deployment.

This report will detail the systematic approach undertaken in designing, modelling, and verifying the smart traffic light controller. It will outline the conceptual framework, the design methodology, the implementation of the controller logic in SystemVerilog, and the simulation procedures employed to validate the functionality of the system. Though a meticulous design and verification process, this assignment aims to deliver a robust solution that significantly improves traffic management at the targeted intersection, setting a precedent for future implementations in similar academic environments.
\subsection{Problem Analysis}
The intersection is briefly described in the assignment handout as an assumed four-way intersection in front of the West gate of Universiti Tunku Abdul Rahman. Since we are to assume an imaginary intersection with little detail given on its traffic flow dynamics, our main goal is set on designing an unbiased but dynamic \ac{tlcs} model for a generic intersection scenario instead of biasing the traffic lights according to differentiations between major roads and side roads. There shall be no specific assumption of traffic load imbalances or the road configuration around the intersection, hence the generic scenario (Figure~\ref{fig:intersection}).
\tikzset{
	trafficlight/.style = {%
			matrix of nodes,
			nodes in empty cells,
			rounded corners,
			draw = black!80,
			fill = black!30,
			scale=0.5,
			nodes = {scale=0.5,circle, minimum size=1pt, anchor=center, draw=black},
			row 1/.style={nodes={fill=red}},
			row 2/.style={nodes={fill=yellow}},
			row 3/.style={nodes={fill=green}},
			row sep=0.5mm,
		}
}
\begin{figure}[H]
	\centering
	\begin{tikzpicture}
		\draw[double =black!50,double distance=1.5cm] (0,-4) -- (0,4)
		(4,0) -- (-4,0);
		\draw[white,line width=1mm, dash pattern=on .3cm off .4cm] (0,-4) -- (0,4)
		(4,0) -- (-4,0);
		\node (Intersection) at (0,0) [rectangle,fill=black!50,minimum height=1.5cm, minimum width=1.5cm,draw=black!50]{};
		\node [rectangle,fill=white,minimum height=1mm, minimum width=.75cm,inner sep=0pt,yshift=.375cm,anchor=south,rotate=90] at (Intersection.west) {};
		\node [rectangle,fill=white,minimum height=.2cm, minimum width=1mm,inner sep=0pt,anchor=south,rotate=90] at (Intersection.west) {};

		\node [rectangle,fill=white,minimum height=1mm, minimum width=.75cm,inner sep=0pt,xshift=.375cm,anchor=south] at (Intersection.north) {};
		\node [rectangle,fill=white,minimum height=.2cm, minimum width=1mm,inner sep=0pt,anchor=south] at (Intersection.north) {};

		\node [rectangle,fill=white,minimum height=1mm, minimum width=.75cm,inner sep=0pt,yshift=-.375cm,anchor=south,rotate=270] at (Intersection.east) {};
		\node [rectangle,fill=white,minimum height=.2cm, minimum width=1mm,inner sep=0pt,anchor=south,rotate=270] at (Intersection.east) {};

		\node [rectangle,fill=white,minimum height=1mm, minimum width=.75cm,inner sep=0pt,xshift=-.375cm,anchor=south,rotate=180] at (Intersection.south) {};
		\node [rectangle,fill=white,minimum height=.2cm, minimum width=1mm,inner sep=0pt,anchor=south,rotate=180] at (Intersection.south) {};

		\node (senA) at (Intersection.north) [ellipse,minimum width=.4cm,minimum height = 1cm,xshift=.375cm,yshift=2mm,draw=blue!70,fill=blue!25,inner sep=0pt,anchor=south]{\footnotesize{$S_A$}};
		\node (senB) at (Intersection.east) [ellipse,minimum width=.4cm,minimum height = 1cm,xshift=2mm,yshift=-.375cm,rotate=270,draw=blue!70,fill=blue!25,inner sep=1pt,anchor=south]{\rotatebox{-270}{\footnotesize{$S_B$}}};
		\node (senC) at (Intersection.south) [ellipse,minimum width=.4cm,minimum height = 1cm,xshift=-.375cm,yshift=-2mm,rotate=180,draw=blue!70,fill=blue!25,inner sep=0pt,anchor=south]{\rotatebox{-180}{\footnotesize{$S_C$}}};
		\node (senD) at (Intersection.west) [ellipse,minimum width=.4cm,minimum height = 1cm,xshift=-2mm,yshift=.375cm,rotate=90,draw=blue!70,fill=blue!25,inner sep=1pt,anchor=south]{\rotatebox{-90}{\footnotesize{$S_D$}}};

		\matrix[trafficlight,label=right:$L_A$] (A) at (Intersection.north east) [anchor=south west,xshift=2mm,yshift=4mm] { \\ \\ \\};
		\draw[line width=2pt,black] (senA.east) -- (A.west);
		\matrix[trafficlight,label=right:$L_B$] (B) at (Intersection.south east) [anchor=north west,xshift=4mm,yshift=-2mm] { \\ \\ \\};
		\draw[line width=2pt,black] (senB.east) -- (B.north);
		\matrix[trafficlight,label=left:$L_C$] (C) at (Intersection.south west) [anchor=north east,xshift=-2mm,yshift=-4mm] { \\ \\ \\};
		\draw[line width=2pt,black] (senC.east) -- (C.east);
		\matrix[trafficlight,label=left:$L_D$] (D) at (Intersection.north west) [anchor=south east,xshift=-4mm,yshift=2mm] { \\ \\ \\};
		\draw[line width=2pt,black] (senD.east) -- (D.south);

		\draw[white,line width=1mm,arrows=-{Stealth[inset=0pt,length=0pt 3]}] (.375cm,3.2) -- node[above=1.2cm,anchor=south,black] {Road $\boldsymbol{A}$} (.375cm,2.5);
		\draw[white,line width=1mm,arrows=-{Stealth[inset=0pt,length=0pt 3]}] (3.2,-.375cm) -- node[right=1.2cm,anchor=west,black] {Road $\boldsymbol{B}$} (2.5,-.375cm);
		\draw[white,line width=1mm,arrows=-{Stealth[inset=0pt,length=0pt 3]}] (-.375cm,-3.2) -- node[below=1.2cm,anchor=north,black] {Road $\boldsymbol{C}$} (-.375cm,-2.5);
		\draw[white,line width=1mm,arrows=-{Stealth[inset=0pt,length=0pt 3]}] (-3.2,.375cm) -- node[left=1.2cm,anchor=east,black] {Road $\boldsymbol{D}$} (-2.5,.375cm);
		\draw[white,line width=1mm,arrows=-{Stealth[inset=0pt,length=0pt 3]}] (-.375cm,2.5) -- (-.375cm,3.2);
		\draw[white,line width=1mm,arrows=-{Stealth[inset=0pt,length=0pt 3]}] (2.5,.375cm) -- (3.2,.375cm);
		\draw[white,line width=1mm,arrows=-{Stealth[inset=0pt,length=0pt 3]}] (.375cm,-2.5) -- (.375cm,-3.2);
		\draw[white,line width=1mm,arrows=-{Stealth[inset=0pt,length=0pt 3]}] (-2.5,-.375cm) -- (-3.2,-.375cm);

		\matrix [draw,anchor=north east] at (current bounding box.north east) {
			\node [label=right:Car sensor] {$S$};    \\
			\node [label=right:Traffic light] {$L$}; \\
		};
	\end{tikzpicture}
	\caption{Generic four-way intersection assumed for our assignment.\label{fig:intersection}}
	\par
\end{figure}
As illustrated in Figure~\ref{fig:intersection}, it is assumed there are four roads labelled $A$, $B$, $C$, and $D$ leading to an intersection, and from each road a driver may head towards the opposite lane of any other road, or possibly even make a U-turn. So each road should need to have its own traffic light to control its traffic flow, and additionally one sensor is placed before the intersection at each road to detect the presence of cars, forming our foundation of interfaces on which to design a dynamic \ac{tlcs}. The sensors had been specified by the assignment handout to output TRUE if there are cars present, and FALSE if otherwise.

\subsection{System Behavior Design Plan\label{section:behavior_plan}}
Due to the assumption that opposite lanes in all directions are accessible from any road at the intersection during its green-light phase, it is implied that throughout the green-light until red-light phase of any particular road, all other roads must stay in the red-light phase to avoid traffic conflict.
\subsubsection{Normal Operation}
When all four roads are busy, the traffic lights should cycle through their phases one by one and update according to their set default (primary) durations of each light phase to keep the timings fair. The default durations for roads $A,B,C,D$ are as follows, and illustrated in Table~\ref{tab:normalop}.
\begin{itemize}
	\item Green-light phase ($G$): 30\unit{\second}
	\item Yellow-light phase ($Y$): 3\unit{\second}
	\item All-Red phase ($R$): 3\unit{\second}
\end{itemize}
% \begin{align}
% 	G_n\xrightarrow{30\unit{\second}}Y_n\xrightarrow{3\unit{\second}}R_n\xrightarrow{3\unit{\second}}G_{n+1}\xrightarrow{30\unit{\second}}\dots,\quad{}n\in\{A,B,C,D\} \label{eq:normalop}
% \end{align}
\begin{table}[H]
	\caption{\acs{tlcs} normal operation sequence generalization.\label{tab:normalop}}
	\centering
	\setlength\tabcolsep{9pt}
	\begin{NiceTabular}[t]{ccccc}
		\toprule
		\multicolumn{4}{c}{Light Status} & \Block{2-1}{Duration (\si\second)}                                \\
		\cmidrule(rl){1-4}
		Road $A$                         & Road $B$                           & Road $C$ & Road $D$ &        \\
		\cmidrule(rl){1-4}\cmidrule(rl){5-5}
		$G$                              & $R$                                & $R$      & $R$      & 30     \\
		$Y$                              & $R$                                & $R$      & $R$      & 3      \\
		$R$                              & $R$                                & $R$      & $R$      & 3      \\
		$R$                              & $G$                                & $R$      & $R$      & 30     \\
		\vdots                           & \vdots                             & \vdots   & \vdots   & \vdots \\
		$R$                              & $R$                                & $R$      & $Y$      & 3      \\
		$R$                              & $R$                                & $R$      & $R$      & 3      \\
		$G$                              & $R$                                & $R$      & $R$      & 30     \\
		\vdots                           & \vdots                             & \vdots   & \vdots   & \vdots \\
		\bottomrule
	\end{NiceTabular}
\end{table}
\subsubsection{Low Car Density Detection}
The green light phase on roads that are detected to be not so busy can be skipped forward after its minimum duration (determined by the low car density timeout duration; 5\si{\second}) to allow traffic for the next road. So, within the entire green-light phase, whenever the sensor at the green-light road has detected no car for 5\unit{\second}, the system shall immediately skip to the yellow-light phase and continue with the normal operation phase durations until the next green-light phase.
\begin{align}
	G_n\xrightarrow[5\unit{\second}\text{ w/ no car on road $n$}]{\text{max }30\unit{\second}}Y_n\xrightarrow{3\unit{\second}}\dots
\end{align}
\subsubsection{Underused Road Detection}
Underused roads that have been empty for a long time can be ignored by the system to minimize redundant green-light phases. At any point during system operation, if a road has had no car detected for at least 30\unit{\second} or since system initialisation, the road is closed off during its red-light phase (then kept in red-light phase) indefinitely or until the next time a car is detected on that road. If there is only one or no road left active with cars passing through, the last road in green-light phase will be stuck in green-light phase indefinitely or until the next time a different road is reopened by a car detection.
