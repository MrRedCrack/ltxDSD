\section{Discussion}
\subsection{\acs{rtl} Synthesis Test Using Quartus Prime Compiler}
An earlier version of the \ac{tlcs} code was modelled at \ac{rtl} and was able to be compiled and run using ModelSim-Intel FPGA Starter Edition 20.1 with no warnings raised. Then, for the sake of verifying that the supposed \ac{rtl} model was demonstrably synthesizable by a computer algorithm, the source code was put to test using Intel Quartus Prime Lite Edition 23.1std.0 Build 991 (latest available verison) to compile for device Cyclone V E Base 5CEBA4F23C7 without \ac{io} pin assignments. The Quartus compiler surprisingly threw a lot of errors and could not finish compiling, which raised doubts about the correctness of the \ac{rtl} design. An investigation found that it was the Quartus software that could not recognize a number of valid syntaxes that should not have impacted synthesis, as listed below:
\begin{itemize}
	\item \texttt{foreach}
	      \begin{itemize}
		      \item Worked around by refactoring the foreach blocks into equivalent for-loop blocks.
	      \end{itemize}
	\item \texttt{timeunit} with second argument (e.g., \texttt{timeunit 100ms \underline{/ 1ms};})
	      \begin{itemize}
		      \item Worked around by removing the second argument and moving its value into a separate \texttt{timeprecision} statement, which has the same meaning.
	      \end{itemize}
	\item \texttt{`begin\_keywords} and \texttt{`end\_keywords} directives
	      \begin{itemize}
		      \item Worked around by commenting out these directives before compiling with Quartus.
	      \end{itemize}
\end{itemize}
Then, there was a \ac{rtl} synthesis weirdness found after making changes to the source code as listed above and being able to view a netlist generated by Quartus, whereby if a variable was declared outside of a for-loop but updated in one of the step assignment statements of the for-loop (e.g., Source Code~\ref{code:syn1}; highlighted line), the variable would not be synthesized by the compiler even if it has an assigned default value and is utilized as an input to another logical operation, causing the expected utilization to vanish altogether without warning in the synthesized result. The problem was fixed by simply taking the variable update statement out of the for-loop initialization statements (e.g., Source Code~\ref{code:syn2}; highlighted line). Nevertheless, there was no difference in functionality when compiled in ModelSim.
\begin{listing}[H]
	\captionof{listing}{Variable \texttt{inc\_offset} updated in a step assignment statement of a for-loop (bad for \acs{rtl} synthesis).\label{code:syn1}}
	\begin{minted}[
    %linenos,
    highlightlines=5,
    highlightcolor=red!10
    ]{systemverilog}
always_comb begin
inc_offset = '0;
for (logic [1:0] i = index + 1'b1, j = '0;
    (j < 3) && (ur_list[i] != '0);
    i++, j++, inc_offset++);
end
\end{minted}
\end{listing}
\begin{code}
	\captionof{listing}{Variable \texttt{inc\_offset} outside of for-loop initialization statements (synthesized fine).\label{code:syn2}}
	\begin{minted}[
    %linenos,
    highlightlines=5,
    highlightcolor=green!10
    ]{systemverilog}
always_comb begin
inc_offset = '0;
for (logic [1:0] i = index + 1'b1, j = '0;
    (j < 3) && (ur_list[i] != '0);
    i++, j++) inc_offset++;
end
\end{minted}
\end{code}
After refactoring the source code to work around the oddities of the Quartus compiler as discussed above, there were no remaining unexpected warnings or errors, and the generated netlist was within expectations and resembled the planned logic block diagram (example shown in Appendix~\ref{app:rtl}). Only the fundamental Lights \ac{fsm} was expressed as a \ac{fsm} block by Quartus (Figure~\ref{fig:lightsfsmquartus}) as it was the only \ac{fsm} that had all of its states explicitly and individually defined, rather than implemented with generalization via arithmetic operations for state transitions.
\begin{figure}[H]
	\centering
	\fbox{\includegraphics[width=.98\textwidth]{lights fsm.png}}
	\caption{Synthesized Lights \acs{fsm} in Quartus.\label{fig:lightsfsmquartus}}
\end{figure}
